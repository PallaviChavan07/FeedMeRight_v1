\chapter{Results}
\label{ch:results}
\section{Content Based Results}
\subsection{Experiment 1 Results}
\label{sec:res1}
\autoref{tb:contentbased}, row one, shows recall, precision and accuracy values for \nameref{sec:cb_ingred_exp}, performed on content-based model by considering only one attribute - ingredients. The values of recall, precision, and accuracy are very low as only one feature is considered. This result can be improved by incorporating another feature as explained in \nameref{sec:cb_ingred_cook_method_exp}.

\subsection{Experiment 2 Results}
\label{sec:res2}
\autoref{tb:contentbased}, row two, shows recall, precision and accuracy values for \nameref{sec:cb_ingred_cook_method_exp}, performed on content-based model by considering two attributes - ingredients and cook methods. Comparing the results of experiments 1 and 2, there is a significant increase in recall, precision, and accuracy due to the addition of cook methods attribute to ingredients. The recall is improved by 42\%, precision is improved by 33\% and accuracy is improved by 33\% when compared to results of experiment 1 with the addition of cook methods feature. \nameref{sec:cb_ingred_cook_method_exp} shows that cook methods aggregate recipes and provides more information about recipes which is a very significant factor to consider in terms of improving performance. The performance of the content-based algorithm can be further improved by incorporating diet-labels attribute as explained in \nameref{sec:cb_ingred_cook_method_diet_label_exp}.

\subsection{Experiment 3 Results}
\label{sec:res3}
\autoref{tb:contentbased}, row three, shows recall, precision and accuracy values for \nameref{sec:cb_ingred_cook_method_diet_label_exp}, performed on content-based model by considering three attributes -  ingredients, cook methods and diet-labels. Comparing the results of experiment 2 and 3, there is a slight increase in recall, precision and accuracy on adding diet-labels attribute.  The recall is improved by approximately 10\%, precision is improved by 9\% and accuracy is improved by 9\%  when compared to results of experiment 2 with the addition of diet labels attribute. Although the addition of diet lables did not add much improvement, it still was helpful in improving the overall performance of content-based algorithm. \nameref{sec:cb_ingred_cook_method_diet_label_exp} shows that, addition of more content or attributes helps in improving the performance of content-based algorithm.

% Please add the following required packages to your document preamble:
% \usepackage[table,xcdraw]{xcolor}
% If you use beamer only pass "xcolor=table" option, i.e. \documentclass[xcolor=table]{beamer}
\begin{table}[H]
\centering
\begin{tabular}{|l|l|l|l|}
\hline
\rowcolor[HTML]{C0C0C0} 
{\color[HTML]{000000} \textbf{Model Name}}                                                                          & {\color[HTML]{000000} \textbf{Recall at 10}} & {\color[HTML]{000000} \textbf{Precision at 10}} & {\color[HTML]{000000} \textbf{Accuracy at 10}} \\ \hline
{\color[HTML]{000000} CB using ingredients}                                                                         & {\color[HTML]{000000} 0.078617241}           & {\color[HTML]{000000} 0.024154169}              & {\color[HTML]{000000} 0.024236065}             \\ \hline
{\color[HTML]{000000} \begin{tabular}[c]{@{}l@{}}CB using ingredients \\ and cook method\end{tabular}}              & {\color[HTML]{000000} 0.105738164}           & {\color[HTML]{000000} 0.032184732}              & {\color[HTML]{000000} 0.032312734}             \\ \hline
{\color[HTML]{000000} \begin{tabular}[c]{@{}l@{}}CB using ingredients, \\ cook method and diet labels\end{tabular}} & {\color[HTML]{000000} 0.111494599}           & {\color[HTML]{000000} 0.035216746}              & {\color[HTML]{000000} 0.035334459}             \\ \hline
\end{tabular}
\caption{Comparison between Evaluation of Content Based Attributes}
\label{tb:contentbased}
\end{table}

\section{Collaborative and Hybrid Results}
\subsection{Experiment 4 Results}
\label{sec:res4}
\autoref{tb:recall}, \autoref{tb:precision} and \autoref{tb:accuracy} row two shows recall, precision and accuracy values for 5, 10 and 20 recommendations, evaluated for \nameref{sec:cf_exp}. Comparing the results of experiment 3 and 4, there is substantial increase in recall, precision and accuracy values for collaborative filtering approach. The recall at 10 has increased from 0.11 to 0.37, precision at 10 has increased from 0.03 to 0.20 and accuracy has increased from 0.03 to 0.16 when compared to results of content-based using all three attributes. The results explains that collaborative filtering using SVD outperforms content-based model for the dataset used in this thesis.

\subsection{Experiment 5 Results}
\label{sec:res5}
\autoref{tb:recall}, \autoref{tb:precision} and \autoref{tb:accuracy} row three shows recall, precision and accuracy values for 5, 10 and 20 recommendations, evaluated for \nameref{sec:hybrid_exp}. Comparing the results of experiments 3, 4 and 5, below are some observations.
\begin{itemize}
\item Significant increase in recall results for recommendations of 5, 10 and 20 across content-based, collaborative and hybrid models. It was expected to improve performance of hybrid model by combining content-based and collaborative model together. On comparing the recall at 10 for content-based and hybrid, we see the significant increase from 0.11 to 0.43. Similarly for collaborative and hybrid, recall at 10 has increased from 0.37 to 0.43. 
\item There is substantial increase in precision of hybrid model when compared to content-based model results for recommendations of 5, 10 and 20. However, surprisingly there is negligible decline in precision of hybrid model compare to the collaborative model results for recommendations of 5, 10 and 20.
\item The accuracy of hybrid model has increased substaintially when compared to the content-based model. There is approximately 12\% increase in accuracy at 5, 10 and 20 recommendations for the hybrid model when compared to the collaborative filtering model. 
\item The performance of hybrid model was expected to be high in terms of recall, precision and accuracy. The hybrid model outperforms traditional approaches in terms of recall and accuracy metric. On the other hand, there is a slight decline in performance of precision compared to collaborative model using SVD. This thesis aims to recommend recipes based on user's taste and preferences where relevancy of recipes is important to the user. Recall refers to the percentage of total relevant results correctly classified. In that case, the trade-off between recall and precision can be accepted. 
\end{itemize}

% Please add the following required packages to your document preamble:
% \usepackage[table,xcdraw]{xcolor}
% If you use beamer only pass "xcolor=table" option, i.e. \documentclass[xcolor=table]{beamer}
\begin{table}[H]
\centering
\begin{tabular}{|l|l|l|l|}
\hline
\rowcolor[HTML]{C0C0C0} 
{\color[HTML]{000000} \textbf{Model Name}}     & {\color[HTML]{000000} \textbf{Recall at 5}} & {\color[HTML]{000000} \textbf{Recall at10}} & {\color[HTML]{000000} \textbf{Recall at 20}} \\ \hline
{\color[HTML]{000000} Content-based}           & {\color[HTML]{000000} 0.082525879}          & {\color[HTML]{000000} 0.111494599}          & {\color[HTML]{000000} 0.14501192}            \\ \hline
{\color[HTML]{000000} Collaborative Filtering} & {\color[HTML]{000000} 0.245029414}          & {\color[HTML]{000000} 0.376785432}          & {\color[HTML]{000000} 0.460967698}           \\ \hline
{\color[HTML]{000000} Hybrid}                  & {\color[HTML]{000000} 0.256281818}          & {\color[HTML]{000000} 0.43370876}           & {\color[HTML]{000000} 0.554458889}           \\ \hline
\end{tabular}
\caption{Recall comparison between CB, CF and Hybrid models}
\label{tb:recall}
\end{table}

\vspace{-0.5em}

% Please add the following required packages to your document preamble:
% \usepackage[table,xcdraw]{xcolor}
% If you use beamer only pass "xcolor=table" option, i.e. \documentclass[xcolor=table]{beamer}
\begin{table}[H]
\centering
\begin{tabular}{|l|l|l|l|}
\hline
\rowcolor[HTML]{C0C0C0} 
{\color[HTML]{000000} \textbf{Model Name}}     & {\color[HTML]{000000} \textbf{Precision at 5}} & {\color[HTML]{000000} \textbf{Precision at10}} & {\color[HTML]{000000} \textbf{Precision at 20}} \\ \hline
{\color[HTML]{000000} Content-based}           & {\color[HTML]{000000} 0.050411997}             & {\color[HTML]{000000} 0.035216746}             & {\color[HTML]{000000} 0.023729464}              \\ \hline
{\color[HTML]{000000} Collaborative Filtering} & {\color[HTML]{000000} 0.20572189}              & {\color[HTML]{000000} 0.204567584}             & {\color[HTML]{000000} 0.20473256}               \\ \hline
{\color[HTML]{000000} Hybrid}                  & {\color[HTML]{000000} 0.204950782}             & {\color[HTML]{000000} 0.203729964}             & {\color[HTML]{000000} 0.203750943}              \\ \hline
\end{tabular}
\caption{Precision comparison between CB, CF and Hybrid models}
\label{tb:precision}
\end{table}

\vspace{-0.5em}

% Please add the following required packages to your document preamble:
% \usepackage[table,xcdraw]{xcolor}
% If you use beamer only pass "xcolor=table" option, i.e. \documentclass[xcolor=table]{beamer}
\begin{table}[H]
\centering
\begin{tabular}{|l|l|l|l|}
\hline
\rowcolor[HTML]{C0C0C0} 
{\color[HTML]{000000} \textbf{Model Name}}     & {\color[HTML]{000000} \textbf{Accuracy at 5}} & {\color[HTML]{000000} \textbf{Accuracy at10}} & {\color[HTML]{000000} \textbf{Accuracy at 20}} \\ \hline
{\color[HTML]{000000} Content-based}           & {\color[HTML]{000000} 0.050483648}            & {\color[HTML]{000000} 0.035334459}            & {\color[HTML]{000000} 0.0238395}               \\ \hline
{\color[HTML]{000000} Collaborative Filtering} & {\color[HTML]{000000} 0.197410308}            & {\color[HTML]{000000} 0.166999335}            & {\color[HTML]{000000} 0.118616613}             \\ \hline
{\color[HTML]{000000} Hybrid}                  & {\color[HTML]{000000} 0.203255028}            & {\color[HTML]{000000} 0.183965403}            & {\color[HTML]{000000} 0.136176365}             \\ \hline
\end{tabular}
\caption{Accuracy comparison between CB, CF and Hybrid models}
\label{tb:accuracy}
\end{table}
