\chapter{Results}
\label{ch:results}
\section{Content Based Results}
\subsection{Experiment 1 Results}
\autoref{tb:contentbased}, row one, shows recall, precision and accuracy values for \nameref{sec:cb_ingred_exp}, performed on content based model by considering only one attribute in ingredients.

\subsection{Experiment 2 Results}
\autoref{tb:contentbased}, row two, shows recall, precision and accuracy values for \nameref{sec:cb_ingred_cook_method_exp}, performed on content-based model by considering two attributes in ingredients and cooking methods. Comparing the results of experiments 1 and 2, there is a significant increase in recall, precision and accuracy due to the addition of cook methods attribute to ingredients.

\subsection{Experiment 3 Results}
\autoref{tb:contentbased}, row three, shows recall, precision and accuracy values for \nameref{sec:cb_ingred_cook_method_diet_label_exp}, performed on content based model by considering three attributes in ingredients, cooking methods and diet-labels. Comparing the results of experiment 2 and 3, there is a slight increase in recall, precision and accuracy on adding diet-labels attribute.

% Please add the following required packages to your document preamble:
% \usepackage[table,xcdraw]{xcolor}
% If you use beamer only pass "xcolor=table" option, i.e. \documentclass[xcolor=table]{beamer}
\begin{table}[]
\begin{tabular}{|l|l|l|l|}
\hline
\rowcolor[HTML]{C0C0C0} 
{\color[HTML]{000000} \textbf{Model Name}}                                                                           & {\color[HTML]{000000} \textbf{Recall at 10}} & {\color[HTML]{000000} \textbf{Precision at 10}} & {\color[HTML]{000000} \textbf{Accuracy at 10}} \\ \hline
{\color[HTML]{000000} CB using ingredients}                                                                          & {\color[HTML]{000000} 0.137357974}           & {\color[HTML]{000000} 0.03820108}               & {\color[HTML]{000000} 0.038263398}             \\ \hline
{\color[HTML]{000000} \begin{tabular}[c]{@{}l@{}}CB using ingredients \\ and cook method\end{tabular}}               & {\color[HTML]{000000} 0.187137273}           & {\color[HTML]{000000} 0.052394308}              & {\color[HTML]{000000} 0.052560507}             \\ \hline
{\color[HTML]{000000} \begin{tabular}[c]{@{}l@{}}CB using ingredients, \\  cook method and diet labels\end{tabular}} & {\color[HTML]{000000} 0.194521619}           & {\color[HTML]{000000} 0.055250337}              & {\color[HTML]{000000} 0.055416192}             \\ \hline
\end{tabular}
\caption{Content Based Attributes}
\label{tb:contentbased}
\end{table}


\section{Collaborative and Hybrid Results}
\subsection{Experiment 4 Results}
\autoref{tb:recall}, \autoref{tb:precision} and \autoref{tb:accuracy} row two shows recall, precision and accuracy values for 5, 10 and 20 recommendations evaluated for \nameref{sec:cf_exp}. Comparing the results of experiment 3 and 4, there is substantial increase in recall, precision and accuracy values for collaborative filtering approach.

\subsection{Experiment 5 Results}
\autoref{tb:recall}, \autoref{tb:precision} and \autoref{tb:accuracy} row three shows recall, precision and accuracy values for 5, 10 and 20 recommendations evaluated for \nameref{sec:hybrid_exp}. Comparing the results of experiments 3, 4 and 5, below are some observations.
\begin{itemize}
\item Significant increase in recall results for recommendations of 5, 10 and 20 across content-based, collaborative and hybrid models
\item Substantial increase in precision of collaborative model when compared to content-based model results for recommendations of 5, 10 and 20.
\item Negligible increase in precision of the hybrid model when compared to collaborative model results for recommendations of 5, 10 and 20.
\item For recommendations of 5 and 10 the accuracy for collaborative sees a considerable increase when compared to the content-based model, while a slight increase in accuracy at 20 recommendations.
\item Good increase in accuracy at 5, 10 and 20 recommendations for the hybrid model when compared to the collaborative filtering model.
\end{itemize}


% Please add the following required packages to your document preamble:
% \usepackage[table,xcdraw]{xcolor}
% If you use beamer only pass "xcolor=table" option, i.e. \documentclass[xcolor=table]{beamer}
\begin{table}[H]
\begin{tabular}{|l|l|l|l|}
\hline
\rowcolor[HTML]{C0C0C0} 
{\color[HTML]{000000} \textbf{Model   Name}}   & {\color[HTML]{000000} \textbf{Recall at 5}} & {\color[HTML]{000000} \textbf{Recall at10}} & {\color[HTML]{000000} \textbf{Recall at 20}} \\ \hline
{\color[HTML]{000000} Content-based}           & {\color[HTML]{000000} 0.144363034}          & {\color[HTML]{000000} 0.194521619}          & {\color[HTML]{000000} 0.246029583}           \\ \hline
{\color[HTML]{000000} Collaborative Filtering} & {\color[HTML]{000000} 0.304749297}          & {\color[HTML]{000000} 0.425609771}          & {\color[HTML]{000000} 0.451710877}           \\ \hline
{\color[HTML]{000000} Hybrid}                  & {\color[HTML]{000000} 0.333330239}          & {\color[HTML]{000000} 0.553225406}          & {\color[HTML]{000000} 0.63603721}            \\ \hline
\end{tabular}
\caption{Recall comparison between CB, CF and Hybrid models}
\label{tb:recall}
\end{table}
\vspace{-2em}
% Please add the following required packages to your document preamble:
% \usepackage[table,xcdraw]{xcolor}
% If you use beamer only pass "xcolor=table" option, i.e. \documentclass[xcolor=table]{beamer}
\begin{table}[H]
\begin{tabular}{|l|l|l|l|}
\hline
\rowcolor[HTML]{C0C0C0} 
{\color[HTML]{000000} \textbf{Model   Name}}   & {\color[HTML]{000000} \textbf{Precision at 5}} & {\color[HTML]{000000} \textbf{Precision at10}} & {\color[HTML]{000000} \textbf{Precision at 20}} \\ \hline
{\color[HTML]{000000} Content-based}           & {\color[HTML]{000000} 0.079879755}             & {\color[HTML]{000000} 0.055250337}             & {\color[HTML]{000000} 0.035456619}              \\ \hline
{\color[HTML]{000000} Collaborative Filtering} & {\color[HTML]{000000} 0.205462838}             & {\color[HTML]{000000} 0.205516395}             & {\color[HTML]{000000} 0.205617723}              \\ \hline
{\color[HTML]{000000} Hybrid}                  & {\color[HTML]{000000} 0.20602087}              & {\color[HTML]{000000} 0.206670344}             & {\color[HTML]{000000} 0.206672018}              \\ \hline
\end{tabular}
\caption{Precision comparison between CB, CF and Hybrid models}
\label{tb:precision}
\end{table}
\vspace{-2em}
% Please add the following required packages to your document preamble:
% \usepackage[table,xcdraw]{xcolor}
% If you use beamer only pass "xcolor=table" option, i.e. \documentclass[xcolor=table]{beamer}
\begin{table}[H]
\begin{tabular}{|l|l|l|l|}
\hline
\rowcolor[HTML]{C0C0C0} 
{\color[HTML]{000000} \textbf{Model   Name}}   & {\color[HTML]{000000} \textbf{Accuracy at 5}} & {\color[HTML]{000000} \textbf{Accuracy at10}} & {\color[HTML]{000000} \textbf{Accuracy at 20}} \\ \hline
{\color[HTML]{000000} Content-based}           & {\color[HTML]{000000} 0.080107806}            & {\color[HTML]{000000} 0.055416192}            & {\color[HTML]{000000} 0.035591376}             \\ \hline
{\color[HTML]{000000} Collaborative Filtering} & {\color[HTML]{000000} 0.186462113}            & {\color[HTML]{000000} 0.136871566}            & {\color[HTML]{000000} 0.074100757}             \\ \hline
{\color[HTML]{000000} Hybrid}                  & {\color[HTML]{000000} 0.201762206}            & {\color[HTML]{000000} 0.174157769}            & {\color[HTML]{000000} 0.104027159}             \\ \hline
\end{tabular}
\caption{Accuracy comparison between CB, CF and Hybrid models}
\label{tb:accuracy}
\end{table}

