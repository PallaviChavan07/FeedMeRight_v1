% Itroduction chapter continued..

\section{Related Work}

There are different ways of how recommendations can be made. One of the simplest way is recommending popular products to the user. But in that case, those recommendation will be same for every user. There are well-known techniques that can make personalized recommendations.
Recommender Systems techniques which can make personalized recommendations are normally classified into three different categories \cite{14}. Content-based Filtering, Collaborative Filtering (CF) and Hybrid Approach.\\
Content-based filtering uses contents of the items to and information related to target user. Collaborative filtering primarily focus on set of users and their relation with items. It does not use the data about items. Collaborative filtering systems collects preferences, and with these preferences predict preference of specific user for target items. Hybrid recommender system combines two or more techniques together \cite{12}. Example, combining content-based and collaborative filtering methods in different ways. Recommender systems help in solving the problem of personalized suggestions for any product.
\\
In recent years web application development has grown. With this development food or recipe sharing applications have emerged. Hence the scope of recommender systems in food domain has increased as it is easy to get user feedback in the form of ratings, reviews or comments in the web applications.
\\
There are many reasons why recommending food or recipes are challenging. One of the reason is changing user's mind towards healthy behavior. Another reason is predicting what people would like to eat because it depends on many factors including region, culture, etc. The various work has done before which includes approaches ranging from content based to collaborative filtering to hybrid approach. One of the traditional approach is content based. It tailors the recommendations to the user's taste. A highly personalized system built by breaking down recipe into it's ingredients and scoring based on the ingredients in recipes which were rated positive \cite{13}. It was built on content-based algorithm. Continuation of this work resulted in considering negatively weighting recipes based on ingredients in recipes \cite{15}. Recent work experiment shows that how can one bring healthiness in recommendations by substituting ingredients \cite{17}. Other than content-based, Collaborative filtering based algorithms have also been proposed. Freyne and Berkovsky experimented nearest neighbor approach (CF - KNN) on ratings. But performance of content-based approach was better. In \cite{15} a recipe recommender implemented using SVD which outperformed content-based and collaborative approach experimented by Freyne and Berkovsky. A New recipe recommender system developed based on matrix factorization which collects ratings and tags in the form of feedback \cite{16}. Continuation of this work resulted in incorporating calorie count \cite{18}. 
These approches performs well in predicting what user may like according to user's taste. But it does not help in changing user's behaviour towards helathy lifestyle. By incorporating calorie count we can restrict user to to choose right food within range of his liking. In this thesis, I present a study of comparative analysis of recommender approaches in food domain and varying recommender approach to recommend a healthy recipes by incorporating calorie count.