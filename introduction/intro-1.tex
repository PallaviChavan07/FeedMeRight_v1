\chapter{Introduction}
\section{Motivation}
The internet is vast network of machines that connects large number of computers together worldwide and allowing them to communicate with one another. The World Wide Web is an information sharing model that is built on top of the internet in which information can be accessed or manipulated easily, which has lead to the dramatic growth in increased usage of internet. One of the results has been the dramatic increase of data and a phenomenon known as BigData.
\\
BigData referes to exponentially increasing data at a high volume, high velocity with variety. This huge amount of data has intrinsic value but provides no utility until it's discovered \cite{2}. With such information overload it is problematic to find exactly what a user is looking for. 
\\
One way to find value in BigData is to deeply analyze it and its interrelated features such as new products, corresponding reviews, ratings and user preferences. Formulating information from raw data is an entire discovery process that requires insightful analysis that would recognize patterns to predict user behaviors to recommend products.
\\
Handling BigData through manual processes is very inefficient. More efficient way of processing such huge amount of data is automating the process of classifying, filtering data of users opinions, features, and preferences in order to understand and predict new set of related products.  Recommender systems are tools that filters information and narrow it down based on user's preferences and helps user to choose which he/she may like. They considers the opinions of communities of users to help each individual in order to understand content of interest from overwhelming information \cite{1}. 
\\ 
Recommender systems can be defined as a tool designed to interact with large and complex information spaces to provide information or items that are relevant to the user \cite{4}.
\\
Today, recommender systems are widely used in variety of applications. Initially it applied for commercial use to analyze data. Amazon is a good example of such an e-commerce website. However, it is now present in several different domains including entertainment, news, books, social tags and some more sophisticated products where personalization is critical such as recipes domain.
According to World Health Organization (WHO) \cite{42}, globally $39\%$ of the adults were over-weighted and $13\%$ were obese in 2016. Overweight and obesity can cause diabetes, blood pressure, heart disease and many other chronic diseases \cite{43}. A well balanced diet plays very important role to improve overall health. With rapid changes in our busy lifestyle, people find it difficult to choose healthy eating option \cite{13}. People tend to move towards options which satisfies their taste or an easy by ignoring the health factor which includes required calories and nutrition. Exploring better dishes in such huge information is very tedious. To solve this problem computer scientists can build systems that can help narrowing down the information considering our health and eating history. This thesis further discusses the different approaches for recipe domain to recommend healthy recipes.
\\
Chapter \nameref{ch:background} provides an overview of the process of building recommendation algorithms, basic recommendation techniques, and evaluation metrics to measure the performance of recommendation engines. Further, setup to run experiments, features extraction process, implementation approaches and experiments are described in chapter \nameref{ch:impl}. Finally results and conclusion are discussed in chapter \nameref{ch:results} and \nameref{ch:conclusion} respectively.
\pagebreak


