\chapter{Introduction}
\section{Motivation}
The internet is a vast network of machines that connects a large number of computers together worldwide and allowing them to communicate with one another. The World Wide Web is an information-sharing model that is built on top of the internet in which information can be accessed or manipulated easily, which has lead to the dramatic growth in increased usage of the internet. One of the results has been the dramatic increase in data and a phenomenon known as BigData.
\\
BigData refers to exponentially increasing data at a high volume, the high velocity with variety. This huge amount of data has intrinsic value but provides no utility until it's discovered \cite{2}. With such information overload, it is problematic to find exactly what a user is looking for.
\\
One way to find value in BigData is to deeply analyze it and its interrelated features such as new products, corresponding reviews, ratings, and user preferences. Formulating information from raw data is an entire discovery process that requires an insightful analysis that would recognize patterns to predict user behaviors to recommend products.
\\
Handling BigData through manual processes is very inefficient. A more efficient way of processing such a huge amount of data is automating the process of classifying, filtering data of users' opinions, features, and preferences in order to understand and predict a new set of related products. Recommender systems are tools that filter information and narrow it down based on a user's preferences and helps a user to choose which he/she may like. They consider the opinions of communities of users to help each individual to understand the content of interest from overwhelming information \cite{1}.
\\
Recommender systems can be defined as a tool designed to interact with large and complex information spaces to provide information or items that are relevant to the user \cite{4}.
\\
Today, recommender systems are widely used in a variety of applications. Initially, it applied for commercial use to analyze data. Amazon is a good example of such an e-commerce website. However, it is now present in several different domains including entertainment, news, books, social tags, and some more sophisticated products where personalization is critical such as recipes domain.
According to the World Health Organization (WHO) \cite{42}, globally $39\%$ of the adults were over-weighted and $13\%$ were obese in 2016. Overweight and obesity can cause diabetes, blood pressure, heart disease and many other chronic diseases \cite{43}. A well-balanced diet plays a very important role to improve overall health. With rapid changes in our busy lifestyle, people find it difficult to choose healthy eating options \cite{13}. People tend to move towards options that satisfy their taste or easy by ignoring the health factor which includes required calories and nutrition. Exploring better dishes in such huge information is very tedious. To solve this problem computer scientists can build systems that can help to narrow down the information considering our health and eating history. This thesis further discusses the different approaches for the recipe domain to recommend healthy recipes.
\\
Chapter \nameref{ch:background} provides an overview of the process of building recommendation algorithms, basic recommendation techniques, and evaluation metrics to measure the performance of recommendation engines. Further, set up to run experiments, features extraction process, implementation approaches and experiments are described in chapter \nameref{ch:impl}. Finally results and conclusion are discussed in chapter \nameref{ch:results} and \nameref{ch:conclusion} respectively.

\section{Contributions}
\begin{itemize}
\item \textbf{Additional Attributes:} In papers \cite{13,15} content-based technique is used based on only ingredients attribute. This thesis considers the content-based technique with the usage of more attributes such as cook methods and diet labels in addition to ingredients.
\item \textbf{Consideration of Health Factor:} The work done in the researches \cite{13,15,16} considers only user preferences either in the form of ratings or tags but this thesis considers user's preferences along with user's health factor in terms of understanding how much calories a user should consume.
\item \textbf{Hybrid Approach along with Calorie Restrictions:} The work in the papers \cite{13,15} uses only a content-based approach on the other hand the work in the paper \cite{16} uses only collaborative filtering approach. Contrary, this thesis considers a hybrid approach that uses both content-based and collaborative approaches combined together with calorie restrictions so that system can consider user's preferences along with a user's health factor.
\end{itemize}
\pagebreak


