% Background chapter continued..

\section{Evaluation Metrics for Recommendation Systems}
\label{sec:eval_metrics}

In this section evaluation metrics that are used to evaluate recommender systems algorithms, are explored. Evaluation can be done in two ways, offline and online evaluation \cite{22,23}. In offline evaluation methods, analysis of collected data is divided into training sets and testing sets in the proportion of 80\% training to 20\% testing. The model of the recommender system is trained on the training dataset and the testing dataset is hidden from the engine. To understand the quality of a recommendation engine, one or more combinations of evaluation metrics are used. There are several methods available to evaluate the performance of the recommender systems \cite{22,24}.
  
\subsection{Recall and Precision}
\label{sec:recall_precision}
Recall and precision are most commonly used metrics to evaluate recommendation engines \cite{25}. These metrics can be explained by a confusion matrix \cite{21} as shown in \autoref{confusion_matrix}. \\
\noindent
TP = True Positive denotes all relevant items are recommended by the system.\\
TN = True Negative denotes all irrelevant items are correctly not recommended by the system.  \\
FP = False Positive denotes all irrelevant items which are
incorrectly recommended by the system\\
FN = False Negative denotes relevant items but system failed to recommend.\\

% Please add the following required packages to your document preamble:
% \usepackage[table,xcdraw]{xcolor}
% If you use beamer only pass "xcolor=table" option, i.e. \documentclass[xcolor=table]{beamer}
\begin{table}[]
\centering 
\begin{tabular}{|l|l|l|}
\hline
\rowcolor[HTML]{C0C0C0} 
\textbf{}  & \textbf{Recommeded} & \textbf{Not Recommended} \\ \hline
Relevant   & TP                  & FN                       \\ \hline
Irrelevant & FP                  & TN                       \\ \hline
\end{tabular}
\caption{Confusion Matrix \cite{21}}
\label{confusion_matrix}
\end{table}
\noindent Based on \autoref{confusion_matrix}, precision is calculated as ratio of the relevant items from recommended items to the number of all recommended items. It is given in \autoref{eq:precision}. 

\begin{equation}
Precision = \frac{TP}{TP + FP} 
\label{eq:precision}
\end{equation}

\noindent Based on \autoref{confusion_matrix}, recall is calculated as the ratio of relevant items from recommended items to the number of all relevant items. It is given in \autoref{eq:recall}. 

\begin{equation}
Recall = \frac{TP}{TP + FN} 
\label{eq:recall}
\end{equation}

Larger value of recall and precision implies better recommendations.

\subsection{Accuracy}
\label{sec:accuracy}
Accuracy is an another evaluatiion metrics for recommender systems. Accuracy is the fraction of correct predictions predicted by a system. It is calculated by equation given in \autoref{eq:accuracy}
\begin{align}
Accuracy = \frac{\textrm{Number of Correct predictions}}{\textrm{Total Number of Predictions}} \\
Accuracy = \frac{TP + TN }{TP + TN + FP + FN}
\label{eq:accuracy}
\end{align}

\subsection{Mean Absolute Error (MAE)}
Mean absolute error used to calculate the average deviation or error generated from predicted ratings and actual ratings \cite{26}.

\begin{equation}
MAE = \frac{1}{n} \sum_{i=1}^{n}{\vert{Predicted_i - Actual_i} \vert}
\label{eq:mae}
\end{equation}
\noindent Where, \\ 
$Predicted_i$ \textsf{ denotes predicted ratings given by user to the item} $i.$ \\
$Actual_i$ \textsf{ denotes actual ratings given by user to the item } $i.$ \\
$n$ \textsf{denotes number of items.}

\noindent With this formula, MAE can calculate the general performance of recommender systems but to compare engines with a different rating scale, we can normalize MAE by dividing by the mean MAE value as shown in \autoref{eq:nmae}. 

\begin{equation}
NMAE = \frac{MAE} {Rating_{max} - Rating_{min}}
\label{eq:nmae}
\end{equation}


\subsection{Root Mean Square Error (RMSE)}
RMSE is a variation of MAE. It also measures the average magnitude of the error. But it puts more weight on large errors as shown in \autoref{eq:rmse}. RMSE can be defined as a square root of the average of squared deviations between predicted ratings and real ratings.
\begin{equation}
RMSE = \sqrt{ \frac{1}{n} \sum_{i=1}^{n}{({Predicted_i - Actual_i} ) ^ {2}}}
\label{eq:rmse}
\end{equation}
\noindent Where, \\
$Predicted_i$ \textsf{ denotes predicted ratings given by user to the item} $i.$ \\
$Actual_i$ \textsf{ denotes actual ratings given by user to the item } $i.$ \\
$n$ \textsf{denotes number of items.}

\noindent Normalized RMSE is given in \autoref{eq:nrmse}
\begin{equation}
NRMSE = \frac{RMSE} {Rating_{max} - Rating_{min}}
\label{eq:nrmse}
\end{equation}



\subsection{Mean Reciprocal Rank}

Mean Reciprocal Rank is one of the rank-aware evaluation metrics.
It is a retrieval measure that calculates the reciprocal of the rank at which the first relevant document was recommended \cite{27}.
From the list of generated recommendations, it finds the rank of the first relevant recommended item and computes the reciprocal of that rank. It can be calculated as shown in \autoref{eq:mrr}

\begin{equation}
MRR(O, U) = \frac{1}{\vert U \vert} \sum_{u \in \epsilon} \frac{1} {k_{u}}
\label{eq:mrr}
\end{equation}
\noindent Where, \\
$k_{u}$ denotes the first relevant recommended item for user u. \\
$U$  denotes the number of users. \\
MRR focuses on the first relevant item of the recommended list. Higher reciprocal rank implies a better recommendation engine.


\subsection{Mean Average Precision at cutoff k (MAP at k)}
MAP at k considers the subset of the list recommended by the system while measuring precision. It does not concern about ordering the items in the list. For each user for each relevant item, it computes the precision of the list through that item. After that it average sub-list precisions. Mathematically it is show in \autoref{eq:map_at_k}
\begin{equation}
MAP_{i} = \frac{1}{\vert R_{i} \vert} \sum_{k=1}^{\vert R_{i} \vert} { P( R_{i}[k]) }
\label{eq:map_at_k}
\end{equation}

\noindent Where, \\
$R_{i}$ denotes rating for item $i$. \\
$k$ denotes rank from 1 to k.\\
$P$ denotes precision of first relevant item.

\noindent Higher mean precision value implies correct recommendations.
