% Background chapter started here..
\chapter{Background}
\label{ch:background}
\section{Related Work}

In recent years web application development has advanced and grown. With this development, food or recipe sharing applications have emerged. Hence the scope of recommender systems in the food domain has increased as it is easy to get user feedback in the form of ratings, reviews, or comments in the web applications.
\\
\noindent There are many reasons why recommending food or recipes are challenging. Such challenge considers changing user's minds towards their own healthy behavior. Another challenge is in predicting what people would like to eat because it depends on many factors including region, culture, and demographic, to name just a few. Prior work in recommender systems includes approaches ranging from content-based systems to collaborative filtering systems to hybrid approaches.
\\

\noindent According to Mika \cite{44}, there are two types of food recommender systems present. The first type considers the user's preferences. It recommends recipes that are similar to the recipes liked by a user in the past. The second type considers only those food items which have been identified by health care providers. It recommends recipes based on nutritional requirements for that user with no consideration of a user's preferences.
\\
\noindent The research of Freyne and Berkovsky \cite{13} uses a content-based technique to predict a rating of an unrated recipe based on the corresponding ingredients included in the recipe. For a user, the framework breaks down unrated recipe into ingredients then calculates a rating of each ingredient based on the rating of all other rated recipes which contain the same ingredients. After calculating ratings for all ingredients of an unseen recipe, the framework predicts the rating of a user for the unseen recipe based on an average of all rating values of all ingredients in this targeted unseen recipe. Recipes with high prediction values will be recommended.
\\
\noindent Continuation of Freyne and Berkovsky's \cite{13} work resulted in considering the weighting factor for ingredients \cite{15}. This framework takes a different approach by using two matrices for user features. One of them is weighted positively for rating values 4 and 5. Weighting factors for rating values 5 and 4 were 2 and 1 respectively. Contrary, another matrix weighted negatively for rating values 1 and 2. Weighting factors for rating values 2 and 1 were 1 and 2 respectively. The matrix with positive values implies strong likes between ingredients. The matrix with negative values implies strong dislikes between ingredients. Based on ingredients similarity scores, recipes were recommended.
\\
\noindent Recent work experiment done by Chun-Yuen Teng, Yu-Ru Lin, and Lada A. Adamic \cite{17} shows that how can one bring a 'healthy' factor in recommendations by substituting ingredients.
\\
\noindent Other than content-based, Collaborative filtering based algorithms have also been proposed. Freyne and Berkovsky experimented with using the nearest neighbor approach (CF - KNN) on ratings. In \cite{15} a recipe recommender implemented using Singular Value Decomposition (SVD) outperformed a content-based and collaborative approach experimented by Freyne and Berkovsky.
\\
\noindent Elahi proposed a food recommender system using active learning algorithm and matrix factorization \cite{16} which collects ratings and tags in the form of feedback. This system provides human-computer interaction for collecting user preferences in the form of ratings and tags. Every user and every recipe are modeled by vectors that represent their latent features. Continuation of this work resulted in incorporating calorie count \cite{18}.
These approaches performed well in predicting what users may like according to user's taste, but it did not help in changing a user's behavior towards a healthy lifestyle. By incorporating calorie count we can restrict a user to choose the right food within range of their personal tastes. In this thesis, I present a comparative analysis of recommender approaches in the food domain and varying recommender approaches to recommend healthy recipes by incorporating calorie count and look to extend the work by \cite{16,18}.