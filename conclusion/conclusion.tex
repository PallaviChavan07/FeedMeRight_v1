\chapter{Conclusion and Future Work}
\label{ch:conclusion}
This thesis aims to recommend healthy recipes by considering the user's calorie intake requirements. Experiments were performed on a large dataset of recipes and users along with unique interactions between users and recipes. \\
In content-based filtering experiment, results showed better performance of recommendations and metrics such as recall, precision and accuracy, when more than one attribute (ingredients, cook method and diet-labels) were considered. \\
Collaborative filtering experiment's performance which was solely based on ratings without considering the complexity of a user's taste showed drastic improvements in recall precision and accuracy.\\
To get the best out of both systems, a hybrid approach was built combining content-based for user's taste and collaborative for a variety of other recipes, to get an overall good performance out of the system.
Metrics from results in section \nameref{ch:results}, shows that even if the performance of the Hybrid approach has increased slightly compared to collaborative filtering, it's more efficient in terms of recommendations due to consideration of user's preference and calorie intake restrictions. \\
\\
As future work, we will consider more features in content-based model such as recipe diversity.
\\Along with calorie balance, we can also account for more nutrition and user's health information such as cholesterol, blood sugar levels, to recommend targeted healthier recipes.