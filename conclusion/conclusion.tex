\chapter{Conclusion and Future Work}
\label{ch:conclusion}
The aim of this thesis is to recommend healthy recipes by considering user's calorie intake requirements. Experiments were performed on large dataset of recipes and users along with unique interactions between users and recipes. \\
In content based filtering experiment, results showed better performance in terms of metrics and recommendations when more than one attribute (ingredients, cook method and diet-labels) were considered. \\
Collaborative filtering experiment which was solely based on ratings without considering complexity of a user's taste performance showed drastic improvements in recall precision and accuracy.
To get the best out of both systems, a hybrid approach was built combining content-based for user's taste and collaborative for variety of other recipes, to get an overall good performance out of the system.
Metrics from results in section ??, shows that even if the performance of Hybrid approach has increased slightly compared to collaborative filtering, its more efficient in terms of recommendations due to consideration of user's preference and calorie intake resstrictions. \\
\\
As future work, we will consider more features in content-based model such as recipe diversity. Along with calorie balance, we can also account for more nutritions and user's heath information such as cholestrol, blood sugar levels, to recommend targated healthier recipes.